\documentclass[a4paper,12pt]{article}
\usepackage[utf8]{inputenc}
\usepackage[ngerman]{babel}
\usepackage{titlesec}

\newcommand{\sectionbreak}{\clearpage}

\title{Antrag auf Festlegung einer Prüfungsordnung für Projektarbeiten im Fach Informatik.}
\author{Paul Jonas Dohle}
\date{\today}


\begin{document}

\maketitle
\thispagestyle{empty} % removes page number
\newpage

\tableofcontents


\section{Einleitung}
Im digitalen Informationszeitalter ist ein breites Verständnis von Computersystemen und deren Funktionsweise von zentraler Bedeutung. Die Fähigkeit, eigene Software zu entwerfen und in einer Programmiersprache umzusetzen, ist dabei eine wesentliche Kompetenz, die SuS erwerben sollten. Dieses Wissen ermöglicht es ihnen, die logischen Prozesse nachzuvollziehen, die im Inneren eines Computers ablaufen, und ein tieferes Verständnis für die Technologien zu entwickeln, die unser Leben prägen.

Ein fundiertes Verständnis dieser Abläufe erleichtert nicht nur die verantwortungsvolle Nutzung digitaler Medien, sondern legt auch die Grundlage für eine kritische Auseinandersetzung mit aktuellen und zukünftigen technologischen Entwicklungen. So wird es den SuS beispielsweise möglich, die Mechanismen und Algorithmen hinter sozialen Medien zu durchschauen, die Funktionsweise generativer KI-Modelle zu verstehen und die Chancen und Risiken des Internets als Ganzes fundiert zu bewerten.

Diese Kenntnisse fördern nicht nur die Medienkompetenz, sondern auch die Fähigkeit, technologiebezogene Entscheidungen bewusst und reflektiert zu treffen. In einer zunehmend digitalisierten Welt ist dies entscheidend, um die vielfältigen Möglichkeiten moderner Technologien sinnvoll zu nutzen, während gleichzeitig potenzielle Gefahren erkannt und minimiert werden können. Letztlich trägt dies dazu bei, die SuS zu selbstbewussten, verantwortungsvollen und kritisch denkenden Teilnehmern der digitalen Gesellschaft zu machen.

Das Wahlpflichtfach Informatik der Klassenstufe 9 und 10 spielt eine entscheidende Rolle, um SuS das notwendige Wissen und die Kompetenzen zu vermitteln, die sie für die selbstbestimmte Teilhabe in der digitalen Welt benötigen. Hier werden die Grundlagen geschaffen, um Computersysteme nicht nur zu verstehen, sondern auch aktiv mit ihnen zu arbeiten. Im Unterricht sollen die SuS lernen, wie digitale Technologien funktionieren, und erlangen Einblicke in Themen wie Algorithmen, Datenstrukturen, Netzwerke und Programmierung. Diese Fähigkeiten sind nicht nur für eine mögliche berufliche Zukunft im Bereich der IT relevant, sondern fördern auch allgemein das logische Denken und die Problemlösungsfähigkeiten.

Ein zentrales Ziel des Informatikunterrichts ist es, den SuS Werkzeuge an die Hand zu geben, mit denen sie digitale Prozesse analysieren, hinterfragen und gestalten können. Dies umfasst das Verständnis von Automatisierung und Datenverarbeitung genauso wie den sicheren Umgang mit Technologien im Alltag. Indem sie selbst Programme schreiben und Systeme entwickeln, erleben die SuS, wie sie eigene Ideen in digitale Lösungen umsetzen können. Dies stärkt ihr technisches Selbstbewusstsein und zeigt, dass sie nicht nur Konsumenten, sondern auch aktive Gestalter der digitalen Welt sein können.

Darüber hinaus leistet der Informatikunterricht einen wichtigen Beitrag zur Förderung der Medienkompetenz. Themen wie Datenschutz, Cyber-Sicherheit und ethische Fragestellungen im Umgang mit KI und sozialen Medien sind feste Bestandteile des Unterrichts. Die SuS lernen, kritisch mit Informationen umzugehen, digitale Inhalte zu bewerten und ihre Privatsphäre zu schützen. Dies befähigt sie, nicht nur technische Probleme zu lösen, sondern auch gesellschaftliche Herausforderungen im Kontext der Digitalisierung zu erkennen und zu bewältigen.


\section{Die Problematik der Unklarheit der Bewerungsrichtlinien}
Im Lehrplan des Europa-Gymnasiums Warstein ist im Fach Informatik in der Klassenstufe 10 eine Unterrichtseinheit zur Programmiersprache Python angesetzt.Dort sollen die SuS an ihre Kenntnisse des vorhergehenden Unterrichtsvorhabens anknüpfen und die spielerisch erlernten Konzepte durch textuelle Programmierung vertiefen. Zum Abschluss dieser Einheit sieht der Lehrplan eine fakultative Projektarbeit vor. Hier können die SuS Ihre Kreativität entfalten und die erlernten Fähigkeiten unter Beweis stellen. Im Gegensatz zu isolierten Beispielaufgaben kann eine Projektarbeit eine große Bandbreite von Konzepten und Prinzipien abdecken.

In der Kürze von lediglich 30 Unterrichtsstunden kann auf viele Dinge selbstverständlich nur oberflächlich eingegangen werden. Besitzen einzelne SuS bereits Vorkenntnisse im genannten Bereich, so lassen sich diese Wissensdifferenzen nicht im besagten Zeitraum überwinden. Somit sind große Disparitäten zwischen Arbeiten einzelner Schüler erwartbar Und müssen als solche hingenommen werden.


\subsection{Funktion von Projektarbeiten}
Basierend auf den Vorstellungen der Lehrkraft und den Rahmenbedingungen des Lehrplans kann die Projektarbeit unterschiedliche Zielsetzungen verfolgen. Wird sie als Äquivalent zu einer Klassenarbeit konzipiert, liegt der Fokus primär auf der bloßen Wissensabfrage. In diesem Fall dient die Projektarbeit dazu, die Fähigkeit der SuS zu prüfen, die im Unterricht vermittelten Inhalte und Befehle korrekt anzuwenden. Die Bewertung konzentriert sich hierbei ausschließlich auf die syntaktische Korrektheit und die Funktionsfähigkeit des erstellten Projektcodes.

Andere Aspekte wie Lesbarkeit des Codes, Qualität der Dokumentation oder die Effizienz der Lösungsansätze bleiben in dieser Betrachtung außen vor. Stattdessen steht im Vordergrund, ob die SuS die zentralen Konzepte der Programmiersprache verstehen und technisch korrekt umsetzen können. Diese Form der Bewertung ermöglicht eine klare und objektive Einschätzung der erlernten Grundlagen und stellt sicher, dass die SuS mit den wichtigsten Bausteinen der Programmierung vertraut sind.\\
\\

Eine alternative Herangehensweise besteht darin, die Projektarbeit als Abbildung des gesamten Softwareentwicklungsprozesses zu gestalten. Dabei wird den SuS bewusst ein großer Gestaltungsspielraum eingeräumt, um realitätsnahe und kreative Ergebnisse zu fördern. Der Fokus liegt nicht nur auf der technischen Umsetzung, sondern auf einem ganzheitlichen Verständnis der Softwareentwicklung, das auch unabhängig von einer spezifischen Programmiersprache anwendbar ist.

In diesem Rahmen geht es nicht nur darum, die Syntax zu beherrschen, sondern vor allem darum, die Fähigkeit zu entwickeln, eigenständig und methodisch Software zu entwerfen und umzusetzen. Von den SuS wird erwartet, dass sie ein geplantes und strukturiertes Vorgehen zeigen, das typische Schritte der Softwareentwicklung wie Anforderungsanalyse, Konzeption, Implementierung und Testen umfasst.

Darüber hinaus fließen in die Bewertung auch qualitative Aspekte der Arbeit ein. Die Lesbarkeit und Wartbarkeit des Codes, die Qualität der Dokumentation sowie die Überlegungen zur Skalierbarkeit oder Erweiterbarkeit der Software spielen hier eine Rolle. Diese Kriterien orientieren sich an den Anforderungen der realen Softwareentwicklung und bereiten die SuS darauf vor, auch komplexere Projekte strukturiert und nachhaltig umzusetzen.

Eine solche Ausrichtung der Projektarbeit zielt darauf ab, nicht nur die technischen Fertigkeiten der SuS zu fördern, sondern ihnen auch ein Verständnis dafür zu vermitteln, welche Bedeutung eine sorgfältige Planung und eine durchdachte Umsetzung für den Erfolg eines Softwareprojekts haben. Sie lernen, dass gute Software nicht nur funktioniert, sondern auch langfristig wartbar, verständlich und anpassbar sein sollte.

Durch diese Herangehensweise werden nicht nur grundlegende Programmierkenntnisse vertieft, sondern auch übergreifende Kompetenzen vermittelt, die die SuS auf weiterführende Herausforderungen in Studium, Beruf oder privaten Projekten vorbereiten.

\end{document}