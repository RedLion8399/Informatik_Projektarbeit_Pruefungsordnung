\documentclass[a4paper,12pt]{article}
\usepackage[utf8]{inputenc}
\usepackage[ngerman]{babel}

\title{Antrag auf Festlegung einer Prüfungsordnung für Projektarbeiten im Fach Informatik.}
\author{Paul Jonas Dohle}
\date{\today}


\begin{document}

\maketitle
\newpage

\tableofcontents
\newpage

\section{Einleitung}
Im digitalen Informationszeitalter ist ein breites Verständnis von Computersystemen und deren Funktionsweise von zentraler Bedeutung. Die Fähigkeit, eigene Software zu entwerfen und in einer Programmiersprache umzusetzen, ist dabei eine wesentliche Kompetenz, die SuS erwerben sollten. Dieses Wissen ermöglicht es ihnen, die logischen Prozesse nachzuvollziehen, die im Inneren eines Computers ablaufen, und ein tieferes Verständnis für die Technologien zu entwickeln, die unser Leben prägen.

Ein fundiertes Verständnis dieser Abläufe erleichtert nicht nur die verantwortungsvolle Nutzung digitaler Medien, sondern legt auch die Grundlage für eine kritische Auseinandersetzung mit aktuellen und zukünftigen technologischen Entwicklungen. So wird es den SuS beispielsweise möglich, die Mechanismen und Algorithmen hinter sozialen Medien zu durchschauen, die Funktionsweise generativer KI-Modelle zu verstehen und die Chancen und Risiken des Internets als Ganzes fundiert zu bewerten.

Diese Kenntnisse fördern nicht nur die Medienkompetenz, sondern auch die Fähigkeit, technologiebezogene Entscheidungen bewusst und reflektiert zu treffen. In einer zunehmend digitalisierten Welt ist dies entscheidend, um die vielfältigen Möglichkeiten moderner Technologien sinnvoll zu nutzen, während gleichzeitig potenzielle Gefahren erkannt und minimiert werden können. Letztlich trägt dies dazu bei, die SuS zu selbstbewussten, verantwortungsvollen und kritisch denkenden Teilnehmern der digitalen Gesellschaft zu machen.

Das Wahlpflichtfach Informatik der Klassenstufe 9 und 10 spielt eine entscheidende Rolle, um SuS das notwendige Wissen und die Kompetenzen zu vermitteln, die sie für die selbstbestimmte Teilhabe in der digitalen Welt benötigen. Hier werden die Grundlagen geschaffen, um Computersysteme nicht nur zu verstehen, sondern auch aktiv mit ihnen zu arbeiten. Im Unterricht sollen die SuS lernen, wie digitale Technologien funktionieren, und erlangen Einblicke in Themen wie Algorithmen, Datenstrukturen, Netzwerke und Programmierung. Diese Fähigkeiten sind nicht nur für eine mögliche berufliche Zukunft im Bereich der IT relevant, sondern fördern auch allgemein das logische Denken und die Problemlösungsfähigkeiten.

Ein zentrales Ziel des Informatikunterrichts ist es, den SuS Werkzeuge an die Hand zu geben, mit denen sie digitale Prozesse analysieren, hinterfragen und gestalten können. Dies umfasst das Verständnis von Automatisierung und Datenverarbeitung genauso wie den sicheren Umgang mit Technologien im Alltag. Indem sie selbst Programme schreiben und Systeme entwickeln, erleben die SuS, wie sie eigene Ideen in digitale Lösungen umsetzen können. Dies stärkt ihr technisches Selbstbewusstsein und zeigt, dass sie nicht nur Konsumenten, sondern auch aktive Gestalter der digitalen Welt sein können.

Darüber hinaus leistet der Informatikunterricht einen wichtigen Beitrag zur Förderung der Medienkompetenz. Themen wie Datenschutz, Cyber-Sicherheit und ethische Fragestellungen im Umgang mit KI und sozialen Medien sind feste Bestandteile des Unterrichts. Die SuS lernen, kritisch mit Informationen umzugehen, digitale Inhalte zu bewerten und ihre Privatsphäre zu schützen. Dies befähigt sie, nicht nur technische Probleme zu lösen, sondern auch gesellschaftliche Herausforderungen im Kontext der Digitalisierung zu erkennen und zu bewältigen.

\end{document}